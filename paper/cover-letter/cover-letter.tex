\documentclass[11pt]{article}
\usepackage[utf8]{inputenc}
\usepackage[T1]{fontenc}
\usepackage{geometry}
\geometry{margin=1in}
\usepackage{hyperref}
\usepackage{xcolor}
\usepackage{parskip}

\pagestyle{empty}

\begin{document}

\begin{flushright}
\textbf{Ritam Pal}\\
Department of Physics\\
Indian Institute of Science Education and Research\\
Pune 411008, India\\
\href{mailto:ritam.pal@students.iiserpune.ac.in}{ritam.pal@students.iiserpune.ac.in}\\[1em]
\today
\end{flushright}

\vspace{1em}

\noindent
Editorial Office\\
\textit{Proceedings of the National Academy of Sciences}\\
500 Fifth Street NW\\
Washington, DC 20001

\vspace{1.5em}

\noindent Dear Editors,

\vspace{0.5em}

We submit our manuscript ``Voter Turnouts Govern Key Electoral Statistics'' for consideration as a Brief Report in PNAS.

\textbf{Key Finding:} We demonstrate that voter turnout---typically treated as a mere indicator of public interest---encodes far richer information: it is sufficient to accurately predict the complete vote distributions of election winners and runners-up. This universal relationship holds across 12 countries spanning multiple decades and across all electoral scales, from large parliamentary constituencies ($\sim 10^6$ voters) down to individual polling booths ($\sim 10^2$ voters).

\textbf{Significance:} Our discovery offers a new quantitative lens for analyzing elections as complex systems. Using the Random Voting Model (RVM) with only two empirical inputs---turnout data and effective number of candidates---we reproduce the observed vote distributions without any country-specific assumptions. This finding is validated through three independent approaches: extensive empirical analysis, RVM simulations, and analytical derivations in the large-turnout limit. Critically, our results suggest a novel forensic tool: systematic deviations from the turnout-predicted distributions could signal electoral irregularities.

\textbf{Relationship to Prior Work:} This extends our recent Phys.\ Rev.\ Lett.\ work [R.\ Pal et al., \textbf{134}, 017401 (2025)] from victory margins to complete vote distributions, revealing turnout as a fundamental determinant of key electoral outcomes.

\textbf{Contributions:} All authors developed the conceptual framework. R.P.\ performed theoretical calculations, data analysis, and simulations. A.K.\ contributed theoretical insights. M.S.S.\ supervised the project. | \textbf{Conflicts:} None. | \textbf{Data:} Publicly available from official election commissions; code available upon publication.

We suggest categories: \textit{Applied Physical Sciences} and \textit{Social Sciences} (Political Sciences).

\vspace{1em}

\noindent Sincerely,\\[1em]
\textbf{Ritam Pal}, on behalf of all authors

\end{document}
