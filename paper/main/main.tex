\documentclass[9pt,twocolumn,twoside]{pnas-new}
% Use the lineno option to display guide line numbers if required.
\usepackage{ulem}
\templatetype{pnasresearcharticle} % Choose template
\articletype{Election}
\newcommand{\red}[1]{\textcolor{red}{#1}}
\newcommand{\blue}[1]{\textcolor{blue}{#1}}

\title{Voter Turnouts Govern Key Electoral Statistics}

\author[a]{Ritam Pal}
\author[b,c]{Aanjaneya Kumar}
\author[a]{M. S. Santhanam}
%\email{ritam.pal@students.iiserpune.ac.in}
%\email{aanjaneya@santafe.edu}
%\email{santh@iiserpune.ac.in}
\affil[a]{Department of Physics, Indian Institute of Science Education and Research, Pune 411008, India.}
\affil[b]{Santa Fe Institute, 1399 Hyde Park Road, Santa Fe, NM 87501, USA}
\affil[c]{High Meadows Environmental Institute, Princeton University, Princeton, NJ, 08544, USA}

\dates{This manuscript was compiled on \today}

% \significancestatement{Authors must submit a 120-word maximum statement about the significance of their research paper written at a level understandable to an undergraduate educated scientist outside their field of speciality. The primary goal of the significance statement is to explain the relevance of the work in broad context to a broad readership. The significance statement appears in the paper itself and is required for all research papers.}

\keywords{elections $|$ turnouts $|$ voteshare $|$ model}
\begin{abstract}
Elections are central institutions in all the democratic societies. They are regarded as complex system due to varied human interactions and interests that shape their outcomes. Quantitative analysis of election data has provided insights into existence of common patterns in elections. In this work, we show that the voter turnout -- the number of people who actually vote on the election day and hence indicates their faith and interest in the process -- contains crucial information that can \sout{help to} accurately predict several key electoral statistics. Using empirical election data from 12 countries spanning multiple decades and random voting model, we demonstrate that the distributions of votes secured by winners and runner-ups are strongly correlated with turnout distributions. The former can be predicted from the knowledge of turnout distribution. This new direction in the quantitative study of elections can provide newer ways to diagnose elections.

\end{abstract}

\begin{document}

\maketitle
\thispagestyle{firststyle}
\ifthenelse{\boolean{shortarticle}}{\ifthenelse{\boolean{singlecolumn}}{\abscontentformatted}{\abscontent}}{}

\firstpage{4}


Free and fair elections play a pivotal role in functioning democracies, ensuring that governing bodies reflect the people’s mandate. Quantitative understanding of election data is essential to safeguard the integrity of electoral process, understand collective decision making by humans, and uncover universal patterns across different decades, countries and contexts.

In the last three decades, aided by the availability of extensive election data, many studies \cite{CosAlmAnd1999, ForCas2007, mantovani2011scaling, ChaMitFor2013, BokSzaVat2018, hosel2019universality} had identified patterns in the distributions of vote shares garnered by candidates \cite{CalCroAnt2015, BurRanGir2016, MorHisNak2019, Kon2017}, victory margins \cite{jacobson1987marginals, mccann1997threatening, mulligan2003empirical, magrino2011computing, xia2012computing, bhattacharyya2021predicting} and voter turnouts \cite{BorBou2010, BorRayBou2012}. However, these patterns exhibit limited universality \cite{CosAlmAnd1999, ForCas2007, ChaMitFor2013}, usually limited to some geographical regions or in some types of elections. Against this backdrop, a recent work by the authors analyzed a large corpus of election data from 34 countries spanning multiple decades and showed that a scaled distribution of the ratio $M/T$, where $M = V_w - V_r$ is the victory margin and $T$ is the turnout, with $V_w$ and $V_r$ denoting the votes secured by the winner and runner-up, exhibits a robust universality independent of the details of the electoral processes \cite{pal2024universal}. Further, turnouts play a fundamental role in driving the victory margins and, in conjunction with Random Voting Model (RVM), can accurately predict the scaled distribution of victory margins \cite{pal2024universal}.

These findings naturally raise the question: Does voter turnouts contain information about other key electoral statistics? Can we recover those statistics using RVM? To explore these questions, we employ the RVM and empirical data from 12 countries to demonstrate that the voter turnout distribution $g(T)$, combined with the effective number of candidates (defined later~\cite{laakso1979effective}), is sufficient to accurately predict two fundamental electoral statistics: the scaled vote distributions of the winner $Q_1(V_w)$ and the runner-up $Q_2(V_r)$. This prediction holds across all electoral scales---from large parliamentary constituencies ($\sim 10^5-10^6$ voters) down to the smallest polling booth levels ($\sim 10^2-10^3$ voters). Thus, irrespective of electoral system or size, voter turnout emerges as a fundamental metric determining key electoral outcomes and offering a new diagnostic for the assessing health of elections.


\section{Materials and methods}
\normalcolwidth

{\it Framework}: An {\it election} happens at all the $N$ electoral units following the first-post-the-past principle \cite{johnston2007politics}. Let the $i$-th electoral unit (constituency, Congressional District, polling booth etc.) have $c_i$ candidates and $n_i$ voters, where $i=1,2, \dots N$. Usually, only a fraction of the voters cast their votes. This is termed the turnout $T_i \le n_i$, and indicates people's interest in the electoral process. Let $V_{i,1}, V_{i,2}, \dots, V_{i,c_i}$ be the votes secured by $c_i$ candidates such that $\sum_{j=1}^{c_i} V_{i,j}=T_i$. The candidate securing most votes, $V_{w} = {\rm max}(V_{i,1}, V_{i,2}, \dots, V_{i,c_i})$ is declared the winner, and the candidate with the second-highest number of votes, $V_{r}$, is the runner-up.



{\it Model:} The random voting model \cite{pal2024universal}, denoted by RVM($T,c$), is an abstraction of the election framework described above. It takes ($T=\{T_1, T_2, \dots , T_N\}, c=\{c_1, c_2, \dots , c_N\}$) as input. Then, the probability that $j$-th candidate in $i$-th electoral unit attracts electors' votes is:
\begin{equation}
    p_{ij} = \frac{w_{ij}}{\sum_{k=1}^{c_i} w_{ik}}, ~~~w_{ij} \sim \mathcal{U}(0, 1), ~~ (j = 1, 2, \cdots c_i).
    \label{eq:prob}
\end{equation}
In this, $\mathcal{U}(0, 1)$ is the uniform distribution. This model captures the statistical features of empirical election data and the universality embedded in the election data \cite{pal2024universal}. 
%In particular, the random voting model with $c_i = 3$ candidates -- RVM($N,T,3$) -- predicts the scaled \emph{margin} distribution remarkably well, irrespective of electoral scales and countries \cite{pal2024universal}. 

{\it Effective Number of Candidates:} In most large elections, many candidates enter the fray but only a few corner most of the votes. To quantify this concentration in $i$-th electoral unit, we use the effective number of candidates (eNoC) defined as \cite{laakso1979effective}:
\begin{equation}
    c_i^{\rm eff} = \frac{1}{\sum_{k=1}^{c_i} (V_{i,k} / T_{i})^2}, \;\;\;\;\;\; i=1,2, \dots, N.
    \label{eq:effc}
\end{equation}
This measure reflects how many candidates are meaningfully competitive. If a single candidate receives all the votes, then $c_i^{\rm eff}=1$. However, if the votes are shared equally among ${\mathcal C}$ candidates, $c_i^{\rm eff}={\mathcal C}$. Thus, Eq. \ref{eq:effc} is a natural way to characterize the eNoC.

% When averaged over all the electoral units and extracting the closest integer value, $\langle c^{\rm eff} \rangle$, gives eNoC for an entire election. 

{\it Data:} Data for national elections of 12 countries and polling booth level data for Canada and India are obtained from []. Basic statistical features of these data sets and other processing details are given in SI.  

\begin{figure}[t]
    \centering
    \includegraphics[width=1\linewidth]{assets/runner_up_vote_data_simulation.pdf}
    \caption{Scaled distribution of the votes $V_r$ received by the runner-up for 12 countries. Empirical election data (circles) agrees with RVM  simulations (dashed line) obtained using turnout distribution $g(T)$ as input.}
    \label{fig:2}
\end{figure}

\section{Information embedded in the voter turnouts}

\begin{figure}[t]
    \centering
    \includegraphics[width=1\linewidth]{assets/winner_vote_data_simulation.pdf}
    \caption{Scaled distribution of the votes $V_w$ received by the winners for 12 countries. Empirical data (circles) agrees with RVM  simulations (dashed line) obtained using turnout distribution $g(T)$ as input.}
    \label{fig:1}
\end{figure}
Scaled vote distributions for winners $\langle V_w \rangle\, Q_1(V_w)$ and runners-up $\langle V_r \rangle\, Q_2(V_r)$ were examined using national election data from 12 countries, each spanning multiple decades. 
Figure \ref{fig:1} displays the vote distribution for winner as a function of scaled winner votes $V_w / \langle V_w \rangle$. Although the profiles of empirical distributions (circles) vary with countries, the overall structure shows a broad support for $V_w / \langle V_w \rangle <1$, and a rapidly decaying tails for $V_w / \langle V_w \rangle >1$, except for Japan and Germany.
Qualitatively similar feature is observed for the runner-up vote distributions plotted against $V_r / \langle V_r \rangle$ in Fig. \ref{fig:2}. Together,  Figs. \ref{fig:2}-\ref{fig:1}capture the representative features of two essential electoral statistics -- the scaled vote distributions of the winners and the runner-ups.

A key insight emerges when these empirical profiles are compared with predictions from RVM. The RVM requires two empirical inputs: ({\it a}) the observed turnout in each electoral unit and, ({\it b}) the corresponding effective number of candidates $c_i^{\rm eff}$, rounded to the closest integer. No other country-specific assumptions are imposed. With these inputs, RVM generates vote totals for all the candidates, from which the winner and runner-up distributions can be constructed. As seen in Figs. \ref{fig:2}-\ref{fig:1}, the RVM simulations closely mimic the empirical distributions for all countries. This agreement indicates a key result: turnout data encodes quantitative information about both $Q_1(V_w)$ and $Q_2(V_r)$, and this is rich enough for the RVM to reproduce the observed profiles without additional assumptions about the underlying details of the political and electoral processes in each country.

To obtain a mathematical basis for the turnout-driven behavior observed in Figs. \ref{fig:2}-\ref{fig:1}, analytical expressions for the vote-share distributions of winners and runners-up are derived by solving the 3-candidate random voter model in the limit of large turnout, $T \gg 1$. Complete derivation and explicit expressions for the analytical forms $Q_1(V_w)$ and $Q_2(V_r)$ are provided in the SI. Figure \ref{fig:3}(a,b) shows empirical winner vote share distributions (circles) superposed on RVM simulation result (dashed line) and analytical curves (black line) for India and the United States. All three curves are in excellent agreement with one another. Figure \ref{fig:3}(c,d) displays similar results for runner-up vote share, and an excellent agreement is observed among data, simulations and theoretical distributions. Thus, consistency across data, simulations, and analytics confirms that the electoral turnout indeed determines the observed electoral statistics.

% Figure \ref{fig:1} displays scaled distribution of the votes obtained by the winner, {\it i.e.},  $\langle V_w \rangle ~Q_1(V_w)$ plotted against $V_w/\langle V_w \rangle$, for empirical national elections data [] drawn from 12 countries. The empirical distribution (circles) is averaged over multiple elections that have taken place over many decades. As seen in the figure, the scaled $Q_1(V_w)$ is not universal and appears different for each country, but mostly has a broad bulk with a sharp fall-off in the tails except for Japan and Germany. To compare this with RVM predictions, the inputs to RVM are; (a) empirical turnout distribution $g(T)$, (b) actual number of constituencies, (c) {\it averaged} effective number of candidates $\langle c^{\rm eff} \rangle$ computed from empirical election data.  The numerically simulated RVM output gives the votes obtained by all the candidates. The winner's vote distribution, shown as dashed lines, mimics the scaled winner distribution (from empirical data) for all the countries. 

% A similar exercise is carried out for the scaled distribution of votes $V_r$ obtained by the runner-up. Figure \ref{fig:2} shows $\langle V_r \rangle ~Q_1(V_r)$ plotted against $V_r/\langle V_r \rangle$, for the same set of 12 countries as in Fig. \ref{fig:1}. The broad features are similar to that seen in Fig. \ref{fig:1} for the winner votes, except in the case of Japan and Germany. The RVM simulation results for runner-up, shown as dashed line, has a good agreement with the empirical data. Remarkably, despite huge differences in electoral size and processes in these` 12 countries, RVM accurately captures the behavior of $Q_1(V_w)$ and $Q_2(V_r)$, both driven by the empirical turnout distribution $g(T)$. This reveals that the turnout data contains much more information than is usually attributed to it.

% To further strengthen the results in Figs. \ref{fig:1}-\ref{fig:2}, RVM is solved to obtain analytical forms for vote distributions. In the limit of large turnout, $T >> 1$, and taking $c_i \approx \langle c^{\rm eff} \rangle=3$ for all $i$ based on election data, we derive distributions $P_w(v_w)$ and $P_r(v_r)$, respectively, for the winner and runner-up vote share denoted by $v_w$ and $v_r$. Then, using empirical voter turnout distribution $g(T)$ applicable to each country, the winner and runner-up distributions, $Q_1(V_w)$ and $Q_2(V_r)$, are numerically computed. The explicit forms for $Q_1(V_w)$ and $Q_2(V_r)$ and the detailed derivation are presented in SI. In Fig. \ref{fig:3}(a,b), for the US and Indian elections, the (scaled) analytical distribution $\langle V_w \rangle ~ Q_1(V_w)$ (solid line) is superimposed on that obtained from empirical data and the RVM simulations. An excellent agreement is observed among all the three. Figure \ref{fig:3}(c,d) displays similar results for the distribution of runner-up votes $\langle V_r \rangle ~ Q_2(V_r)$.  An excellent agreement shows that analytical forms are consistent with election data and RVM simulations. Though the profiles of scaled $Q_1(V_w)$ and $Q_2(V_r)$ are quite different for each country, the analytical result with $g(T)$ as input closely follows the profiles (Figs. \ref{fig:3}). This confirms that turnouts contain quantitative information about crucial election metrics.


\begin{figure}[t]
    \centering
    \includegraphics[width=0.9\linewidth]{assets/combined_winner_runner_india_us.pdf}
    \caption{Scaled distribution of the votes received by the winner and the runner-up. The analytical prediction (solid line) closely agrees with the RVM simulations (dashed line) and the empirical distributions (circles) for India and the USA.}
    \label{fig:3}
\end{figure}

Do these results in Figs. \ref{fig:1}-\ref{fig:3}, obtained for national elections, hold good at smaller electoral scales? 
To answer this question, we use parliamentary constituency (PC) and polling booth (PB) level data from India and Canada. In Canada, the average electorate size in PC and PB is  $\sim 10^4$ and $10^2$, while the numbers for India are $10^5$ and $10^2$ -- both displaying a clear separation of scales in terms of electorate size.
Recall that Fig. \ref{fig:1}-\ref{fig:2}(a,c) shows scaled distribution of winner and runner-up votes in PC level data, respectively, for India and Canada, and they are in agreement with RVM simulations. Remarkably, at PB levels too, in Fig. \ref{fig:4}(a,c) for India and in Fig. \ref{fig:4}(b,d) for Canada, a similar agreement is observed between the empirical distribution and RVM simulations with empirical turnout data as input. This reinforces the central result that, in fair elections, turnouts govern the statistics of winner and runner-up votes, and this feature is valid at all electoral scales.

In summary, election datasets provide insights into collective decision making by the people. The voter turnout is usually interpreted as indicating the trust and interest in the electoral processes. Going beyond this convention, we have demonstrated that the voter turnout distribution $g(T)$ encodes significant information about several crucial election statistics, namely, about (scaled) distribution of votes received by winner and runner-up. We demonstrate this using empirical data from 12 countries for multiple elections held over many decades. Further evidence comes from theuse of Random Voting Model, which takes empirical turnout data and an effective number of candidates as input, and correctly predicts the empirically observed distributions. This is further strengthened by analytical solution to RVM in excellent agreement with observed data. Remarkably, all these results hold good at all electoral scales -- from large constituencies/Congressional District down to small polling booth levels. Our results point to a new direction in the quantitative study of elections: electoral turnout data contains much more information than is usually attributed to it, and can provide new ways to diagnose elections. Voter turnout is not just a statistical curiosity of public interest, but an indicator of the health of fair elections.  Hence, election oversight bodies across the world must report turnout data at all levels to help analyze the fairness of the process.

%The voter turnout distribution $g(T)$ is a good indicator of the public trust and interest in the electoral process. We show that $g(T)$ encodes information about several crucial election statistics. It contains quantitative information about (scaled) distribution of votes, $Q_1(V_w)$ and $Q_2(V_r)$, received by winner and runner-up in fair elections. We demonstrate this using empirical data from 12 countries of multiple elections held over many decades. To confirm this behavior, we employ the Random Voting Model, which takes empirical turnout data and an effective number of candidates as input, and correctly reproduces the empirically observed distributions $Q_1(V_w)$ and $Q_2(V_r)$. This is further strengthened by analytical solution to RVM, which also agrees with $Q_1(V_w)$ and $Q2(V_r)$. Remarkably, all these results hold good at smaller electoral scales --  polling booth level data considered as a basic electoral unit. These results point to new direction in the quantitative study of elections: electoral turnout data contains much more information than is usually attributed to it, and it can provide new ways to diagnose elections.

\begin{figure}[t]
    \centering
    \includegraphics[width=0.9\linewidth]{assets/india_canada_poll.pdf}
    \caption{Scaled distribution of the votes received by the winner and the runner-up in polling level data for India and Canada. The empirical data (circles) closely agree with the RVM simulations (dashed line) that takes turnout data as input.}
    \label{fig:4}
\end{figure}

\acknow{R.P. and A.K. thank the Prime Minister's Research Fellowship of the Government of India for financial support. The authors acknowledge the National Supercomputing Mission for the use of PARAM Brahma at IISER Pune.}
\showacknow

%apsrev4-2.bst 2019-01-14 (MD) hand-edited version of apsrev4-1.bst
%Control: key (0)
%Control: author (8) initials jnrlst
%Control: editor formatted (1) identically to author
%Control: production of article title (0) allowed
%Control: page (0) single
%Control: year (1) truncated
%Control: production of eprint (0) enabled
\begin{thebibliography}{45}%
\makeatletter
\providecommand \@ifxundefined [1]{%
 \@ifx{#1\undefined}
}%
\providecommand \@ifnum [1]{%
 \ifnum #1\expandafter \@firstoftwo
 \else \expandafter \@secondoftwo
 \fi
}%
\providecommand \@ifx [1]{%
 \ifx #1\expandafter \@firstoftwo
 \else \expandafter \@secondoftwo
 \fi
}%
\providecommand \natexlab [1]{#1}%
\providecommand \enquote  [1]{``#1''}%
\providecommand \bibnamefont  [1]{#1}%
\providecommand \bibfnamefont [1]{#1}%
\providecommand \citenamefont [1]{#1}%
\providecommand \href@noop [0]{\@secondoftwo}%
\providecommand \href [0]{\begingroup \@sanitize@url \@href}%
\providecommand \@href[1]{\@@startlink{#1}\@@href}%
\providecommand \@@href[1]{\endgroup#1\@@endlink}%
\providecommand \@sanitize@url [0]{\catcode `\\12\catcode `\$12\catcode `\&12\catcode `\#12\catcode `\^12\catcode `\_12\catcode `\%12\relax}%
\providecommand \@@startlink[1]{}%
\providecommand \@@endlink[0]{}%
\providecommand \url  [0]{\begingroup\@sanitize@url \@url }%
\providecommand \@url [1]{\endgroup\@href {#1}{\urlprefix }}%
\providecommand \urlprefix  [0]{URL }%
\providecommand \Eprint [0]{\href }%
\providecommand \doibase [0]{https://doi.org/}%
\providecommand \selectlanguage [0]{\@gobble}%
\providecommand \bibinfo  [0]{\@secondoftwo}%
\providecommand \bibfield  [0]{\@secondoftwo}%
\providecommand \translation [1]{[#1]}%
\providecommand \BibitemOpen [0]{}%
\providecommand \bibitemStop [0]{}%
\providecommand \bibitemNoStop [0]{.\EOS\space}%
\providecommand \EOS [0]{\spacefactor3000\relax}%
\providecommand \BibitemShut  [1]{\csname bibitem#1\endcsname}%
\let\auto@bib@innerbib\@empty
%</preamble>
% \bibitem [{\citenamefont {Galam}(1999)}]{galam1999application}%
%   \BibitemOpen
%   \bibfield  {author} {\bibinfo {author} {\bibfnamefont {S.}~\bibnamefont {Galam}},\ }\bibfield  {title} {\bibinfo {title} {Application of statistical physics to politics},\ }\href@noop {} {\bibfield  {journal} {\bibinfo  {journal} {Physica A: Statistical mechanics and its applications}\ }\textbf {\bibinfo {volume} {274}},\ \bibinfo {pages} {132} (\bibinfo {year} {1999})}\BibitemShut {NoStop}%
% \bibitem [{\citenamefont {Gelman}\ \emph {et~al.}(2002)\citenamefont {Gelman}, \citenamefont {Katz},\ and\ \citenamefont {Tuerlinckx}}]{gelman2002mathematics}%
%   \BibitemOpen
%   \bibfield  {author} {\bibinfo {author} {\bibfnamefont {A.}~\bibnamefont {Gelman}}, \bibinfo {author} {\bibfnamefont {J.~N.}\ \bibnamefont {Katz}},\ and\ \bibinfo {author} {\bibfnamefont {F.}~\bibnamefont {Tuerlinckx}},\ }\bibfield  {title} {\bibinfo {title} {The mathematics and statistics of voting power},\ }\href@noop {} {\bibfield  {journal} {\bibinfo  {journal} {Statistical Science}\ ,\ \bibinfo {pages} {420}} (\bibinfo {year} {2002})}\BibitemShut {NoStop}%
% \bibitem [{\citenamefont {Brams}(2008)}]{brams2008}%
%   \BibitemOpen
%   \bibfield  {author} {\bibinfo {author} {\bibfnamefont {S.~J.}\ \bibnamefont {Brams}},\ }\href@noop {} {\emph {\bibinfo {title} {Mathematics and Democracy : Designing better voting and fair-division procedures}}}\ (\bibinfo  {publisher} {Princeton University Press, Princeton},\ \bibinfo {year} {2008})\BibitemShut {NoStop}%
% \bibitem [{\citenamefont {Castellano}\ \emph {et~al.}(2009)\citenamefont {Castellano}, \citenamefont {Fortunato},\ and\ \citenamefont {Loreto}}]{CasForLor2009}%
%   \BibitemOpen
%   \bibfield  {author} {\bibinfo {author} {\bibfnamefont {C.}~\bibnamefont {Castellano}}, \bibinfo {author} {\bibfnamefont {S.}~\bibnamefont {Fortunato}},\ and\ \bibinfo {author} {\bibfnamefont {V.}~\bibnamefont {Loreto}},\ }\bibfield  {title} {\bibinfo {title} {Statistical physics of social dynamics},\ }\href {https://doi.org/10.1103/RevModPhys.81.591} {\bibfield  {journal} {\bibinfo  {journal} {Rev. Mod. Phys.}\ }\textbf {\bibinfo {volume} {81}},\ \bibinfo {pages} {591} (\bibinfo {year} {2009})}\BibitemShut {NoStop}%
% \bibitem [{\citenamefont {Galam}(2012)}]{galam2012}%
%   \BibitemOpen
%   \bibfield  {author} {\bibinfo {author} {\bibfnamefont {S.}~\bibnamefont {Galam}},\ }\href {https://doi.org/https://doi.org/10.1007/978-1-4614-2032-3} {\emph {\bibinfo {title} {Sociophysics: A Physicist's Modeling of Psycho-political Phenomena}}}\ (\bibinfo  {publisher} {Springer New York, NY},\ \bibinfo {year} {2012})\BibitemShut {NoStop}%
% \bibitem [{\citenamefont {Fortunato}(2013)}]{ForMacRed2013}%
%   \BibitemOpen
%   \bibfield  {author} {\bibinfo {author} {\bibfnamefont {S.}~\bibnamefont {Fortunato}},\ }\bibfield  {title} {\bibinfo {title} {Santo fortunato{\textperiodcentered} michael macy{\textperiodcentered} sidney redner},\ }\href@noop {} {\bibfield  {journal} {\bibinfo  {journal} {J Stat Phys}\ }\textbf {\bibinfo {volume} {151}},\ \bibinfo {pages} {1} (\bibinfo {year} {2013})}\BibitemShut {NoStop}%
% \bibitem [{\citenamefont {Sen}\ and\ \citenamefont {Chakrabarti}(2014)}]{SenCha2014}%
%   \BibitemOpen
%   \bibfield  {author} {\bibinfo {author} {\bibfnamefont {P.}~\bibnamefont {Sen}}\ and\ \bibinfo {author} {\bibfnamefont {B.~K.}\ \bibnamefont {Chakrabarti}},\ }\href@noop {} {\emph {\bibinfo {title} {Sociophysics: an introduction}}}\ (\bibinfo  {publisher} {OUP, Oxford},\ \bibinfo {year} {2014})\BibitemShut {NoStop}%
% \bibitem [{\citenamefont {Fern\'andez-Gracia}\ \emph {et~al.}(2014)\citenamefont {Fern\'andez-Gracia}, \citenamefont {Suchecki}, \citenamefont {Ramasco}, \citenamefont {San~Miguel},\ and\ \citenamefont {Egu\'{\i}luz}}]{FerSucRam2014}%
%   \BibitemOpen
%   \bibfield  {author} {\bibinfo {author} {\bibfnamefont {J.}~\bibnamefont {Fern\'andez-Gracia}}, \bibinfo {author} {\bibfnamefont {K.}~\bibnamefont {Suchecki}}, \bibinfo {author} {\bibfnamefont {J.~J.}\ \bibnamefont {Ramasco}}, \bibinfo {author} {\bibfnamefont {M.}~\bibnamefont {San~Miguel}},\ and\ \bibinfo {author} {\bibfnamefont {V.~M.}\ \bibnamefont {Egu\'{\i}luz}},\ }\bibfield  {title} {\bibinfo {title} {Is the voter model a model for voters?},\ }\href {https://doi.org/10.1103/PhysRevLett.112.158701} {\bibfield  {journal} {\bibinfo  {journal} {Phys. Rev. Lett.}\ }\textbf {\bibinfo {volume} {112}},\ \bibinfo {pages} {158701} (\bibinfo {year} {2014})}\BibitemShut {NoStop}%
% \bibitem [{\citenamefont {Braha}\ and\ \citenamefont {de~Aguiar}(2017)}]{BraDeA2017}%
%   \BibitemOpen
%   \bibfield  {author} {\bibinfo {author} {\bibfnamefont {D.}~\bibnamefont {Braha}}\ and\ \bibinfo {author} {\bibfnamefont {M.~A.~M.}\ \bibnamefont {de~Aguiar}},\ }\bibfield  {title} {\bibinfo {title} {Voting contagion: Modeling and analysis of a century of u.s. presidential elections},\ }\href {https://doi.org/10.1371/journal.pone.0177970} {\bibfield  {journal} {\bibinfo  {journal} {PLOS ONE}\ }\textbf {\bibinfo {volume} {12}},\ \bibinfo {pages} {1} (\bibinfo {year} {2017})}\BibitemShut {NoStop}%
% \bibitem [{\citenamefont {Kononovicius}(2019)}]{Kon2019}%
%   \BibitemOpen
%   \bibfield  {author} {\bibinfo {author} {\bibfnamefont {A.}~\bibnamefont {Kononovicius}},\ }\bibfield  {title} {\bibinfo {title} {Compartmental voter model},\ }\href {https://doi.org/10.1088/1742-5468/ab409b} {\bibfield  {journal} {\bibinfo  {journal} {Journal of Statistical Mechanics: Theory and Experiment}\ }\textbf {\bibinfo {volume} {2019}},\ \bibinfo {pages} {103402} (\bibinfo {year} {2019})}\BibitemShut {NoStop}%
% \bibitem [{\citenamefont {Redner}(2019)}]{redner2019reality}%
%   \BibitemOpen
%   \bibfield  {author} {\bibinfo {author} {\bibfnamefont {S.}~\bibnamefont {Redner}},\ }\bibfield  {title} {\bibinfo {title} {Reality-inspired voter models: A mini-review},\ }\href@noop {} {\bibfield  {journal} {\bibinfo  {journal} {Comptes Rendus Physique}\ }\textbf {\bibinfo {volume} {20}},\ \bibinfo {pages} {275} (\bibinfo {year} {2019})}\BibitemShut {NoStop}%
% \bibitem [{\citenamefont {San~Miguel}\ and\ \citenamefont {Toral}(2020)}]{MigTor2020}%
%   \BibitemOpen
%   \bibfield  {author} {\bibinfo {author} {\bibfnamefont {M.}~\bibnamefont {San~Miguel}}\ and\ \bibinfo {author} {\bibfnamefont {R.}~\bibnamefont {Toral}},\ }\bibfield  {title} {\bibinfo {title} {Introduction to the chaos focus issue on the dynamics of social systems},\ }\href@noop {} {\bibfield  {journal} {\bibinfo  {journal} {Chaos: An Interdisciplinary Journal of Nonlinear Science}\ }\textbf {\bibinfo {volume} {30}} (\bibinfo {year} {2020})},\ \bibinfo {note} {see all the papers that are part of this special issue}\BibitemShut {NoStop}%
\bibitem [{\citenamefont {Filho}\ \emph {et~al.}(1999)\citenamefont {Filho}, \citenamefont {Almeida}, \citenamefont {Andrade},\ and\ \citenamefont {Moreira}}]{CosAlmAnd1999}%
  \BibitemOpen
  \bibfield  {author} {\bibinfo {author} {\bibfnamefont {R.~N.~C.}\ \bibnamefont {Filho}}, \bibinfo {author} {\bibfnamefont {M.~P.}\ \bibnamefont {Almeida}}, \bibinfo {author} {\bibfnamefont {J.~S.}\ \bibnamefont {Andrade}},\ and\ \bibinfo {author} {\bibfnamefont {J.~E.}\ \bibnamefont {Moreira}},\ }\bibfield  {title} {\bibinfo {title} {Scaling behavior in a proportional voting process},\ }\href {https://doi.org/10.1103/PhysRevE.60.1067} {\bibfield  {journal} {\bibinfo  {journal} {Phys. Rev. E}\ }\textbf {\bibinfo {volume} {60}},\ \bibinfo {pages} {1067} (\bibinfo {year} {1999})}\BibitemShut {NoStop}%
\bibitem [{\citenamefont {Fortunato}\ and\ \citenamefont {Castellano}(2007)}]{ForCas2007}%
  \BibitemOpen
  \bibfield  {author} {\bibinfo {author} {\bibfnamefont {S.}~\bibnamefont {Fortunato}}\ and\ \bibinfo {author} {\bibfnamefont {C.}~\bibnamefont {Castellano}},\ }\bibfield  {title} {\bibinfo {title} {Scaling and universality in proportional elections},\ }\href {https://doi.org/10.1103/PhysRevLett.99.138701} {\bibfield  {journal} {\bibinfo  {journal} {Phys. Rev. Lett.}\ }\textbf {\bibinfo {volume} {99}},\ \bibinfo {pages} {138701} (\bibinfo {year} {2007})}\BibitemShut {NoStop}%
\bibitem [{\citenamefont {Mantovani}\ \emph {et~al.}(2011)\citenamefont {Mantovani}, \citenamefont {Ribeiro}, \citenamefont {Moro}, \citenamefont {Picoli},\ and\ \citenamefont {Mendes}}]{mantovani2011scaling}%
  \BibitemOpen
  \bibfield  {author} {\bibinfo {author} {\bibfnamefont {M.}~\bibnamefont {Mantovani}}, \bibinfo {author} {\bibfnamefont {H.}~\bibnamefont {Ribeiro}}, \bibinfo {author} {\bibfnamefont {M.}~\bibnamefont {Moro}}, \bibinfo {author} {\bibfnamefont {S.}~\bibnamefont {Picoli}},\ and\ \bibinfo {author} {\bibfnamefont {R.}~\bibnamefont {Mendes}},\ }\bibfield  {title} {\bibinfo {title} {Scaling laws and universality in the choice of election candidates},\ }\href@noop {} {\bibfield  {journal} {\bibinfo  {journal} {Europhysics Letters}\ }\textbf {\bibinfo {volume} {96}},\ \bibinfo {pages} {48001} (\bibinfo {year} {2011})}\BibitemShut {NoStop}%
\bibitem [{\citenamefont {Chatterjee}\ \emph {et~al.}(2013)\citenamefont {Chatterjee}, \citenamefont {Mitrovi{\'c}},\ and\ \citenamefont {Fortunato}}]{ChaMitFor2013}%
  \BibitemOpen
  \bibfield  {author} {\bibinfo {author} {\bibfnamefont {A.}~\bibnamefont {Chatterjee}}, \bibinfo {author} {\bibfnamefont {M.}~\bibnamefont {Mitrovi{\'c}}},\ and\ \bibinfo {author} {\bibfnamefont {S.}~\bibnamefont {Fortunato}},\ }\bibfield  {title} {\bibinfo {title} {Universality in voting behavior: an empirical analysis},\ }\href@noop {} {\bibfield  {journal} {\bibinfo  {journal} {Scientific reports}\ }\textbf {\bibinfo {volume} {3}},\ \bibinfo {pages} {1049} (\bibinfo {year} {2013})}\BibitemShut {NoStop}%
\bibitem [{\citenamefont {Bokányi}\ \emph {et~al.}(2018)\citenamefont {Bokányi}, \citenamefont {Szállási},\ and\ \citenamefont {Vattay}}]{BokSzaVat2018}%
  \BibitemOpen
  \bibfield  {author} {\bibinfo {author} {\bibfnamefont {E.}~\bibnamefont {Bokányi}}, \bibinfo {author} {\bibfnamefont {Z.}~\bibnamefont {Szállási}},\ and\ \bibinfo {author} {\bibfnamefont {G.}~\bibnamefont {Vattay}},\ }\bibfield  {title} {\bibinfo {title} {Universal scaling laws in metro area election results},\ }\href {https://doi.org/10.1371/journal.pone.0192913} {\bibfield  {journal} {\bibinfo  {journal} {PLOS ONE}\ }\textbf {\bibinfo {volume} {13}},\ \bibinfo {pages} {1} (\bibinfo {year} {2018})}\BibitemShut {NoStop}%
\bibitem [{\citenamefont {H{\"o}sel}\ \emph {et~al.}(2019)\citenamefont {H{\"o}sel}, \citenamefont {M{\"u}ller},\ and\ \citenamefont {Tellier}}]{hosel2019universality}%
  \BibitemOpen
  \bibfield  {author} {\bibinfo {author} {\bibfnamefont {V.}~\bibnamefont {H{\"o}sel}}, \bibinfo {author} {\bibfnamefont {J.}~\bibnamefont {M{\"u}ller}},\ and\ \bibinfo {author} {\bibfnamefont {A.}~\bibnamefont {Tellier}},\ }\bibfield  {title} {\bibinfo {title} {Universality of neutral models: Decision process in politics},\ }\href@noop {} {\bibfield  {journal} {\bibinfo  {journal} {Palgrave Communications}\ }\textbf {\bibinfo {volume} {5}},\ \bibinfo {pages} {1} (\bibinfo {year} {2019})}\BibitemShut {NoStop}%
\bibitem [{\citenamefont {Calvão}\ \emph {et~al.}(2015)\citenamefont {Calvão}, \citenamefont {Crokidakis},\ and\ \citenamefont {Anteneodo}}]{CalCroAnt2015}%
  \BibitemOpen
  \bibfield  {author} {\bibinfo {author} {\bibfnamefont {A.~M.}\ \bibnamefont {Calvão}}, \bibinfo {author} {\bibfnamefont {N.}~\bibnamefont {Crokidakis}},\ and\ \bibinfo {author} {\bibfnamefont {C.}~\bibnamefont {Anteneodo}},\ }\bibfield  {title} {\bibinfo {title} {Stylized facts in brazilian vote distributions},\ }\href {https://doi.org/10.1371/journal.pone.0137732} {\bibfield  {journal} {\bibinfo  {journal} {PLOS ONE}\ }\textbf {\bibinfo {volume} {10}},\ \bibinfo {pages} {1} (\bibinfo {year} {2015})}\BibitemShut {NoStop}%
\bibitem [{\citenamefont {Burghardt}\ \emph {et~al.}(2016)\citenamefont {Burghardt}, \citenamefont {Rand},\ and\ \citenamefont {Girvan}}]{BurRanGir2016}%
  \BibitemOpen
  \bibfield  {author} {\bibinfo {author} {\bibfnamefont {K.}~\bibnamefont {Burghardt}}, \bibinfo {author} {\bibfnamefont {W.}~\bibnamefont {Rand}},\ and\ \bibinfo {author} {\bibfnamefont {M.}~\bibnamefont {Girvan}},\ }\bibfield  {title} {\bibinfo {title} {Competing opinions and stubborness: Connecting models to data},\ }\href {https://doi.org/10.1103/PhysRevE.93.032305} {\bibfield  {journal} {\bibinfo  {journal} {Phys. Rev. E}\ }\textbf {\bibinfo {volume} {93}},\ \bibinfo {pages} {032305} (\bibinfo {year} {2016})}\BibitemShut {NoStop}%
\bibitem [{\citenamefont {Mori}\ \emph {et~al.}(2019)\citenamefont {Mori}, \citenamefont {Hisakado},\ and\ \citenamefont {Nakayama}}]{MorHisNak2019}%
  \BibitemOpen
  \bibfield  {author} {\bibinfo {author} {\bibfnamefont {S.}~\bibnamefont {Mori}}, \bibinfo {author} {\bibfnamefont {M.}~\bibnamefont {Hisakado}},\ and\ \bibinfo {author} {\bibfnamefont {K.}~\bibnamefont {Nakayama}},\ }\bibfield  {title} {\bibinfo {title} {Voter model on networks and the multivariate beta distribution},\ }\href {https://doi.org/10.1103/PhysRevE.99.052307} {\bibfield  {journal} {\bibinfo  {journal} {Phys. Rev. E}\ }\textbf {\bibinfo {volume} {99}},\ \bibinfo {pages} {052307} (\bibinfo {year} {2019})}\BibitemShut {NoStop}%
\bibitem [{\citenamefont {Kononovicius}(2018)}]{Kon2017}%
  \BibitemOpen
  \bibfield  {author} {\bibinfo {author} {\bibfnamefont {A.}~\bibnamefont {Kononovicius}},\ }\bibfield  {title} {\bibinfo {title} {Modeling of the parties' vote share distributions},\ }\href@noop {} {\bibfield  {journal} {\bibinfo  {journal} {Acta Physica Polonica A}\ }\textbf {\bibinfo {volume} {133}},\ \bibinfo {pages} {1450} (\bibinfo {year} {2018})}\BibitemShut {NoStop}%
\bibitem [{\citenamefont {Borghesi}\ and\ \citenamefont {Bouchaud}(2010)}]{BorBou2010}%
  \BibitemOpen
  \bibfield  {author} {\bibinfo {author} {\bibfnamefont {C.}~\bibnamefont {Borghesi}}\ and\ \bibinfo {author} {\bibfnamefont {J.-P.}\ \bibnamefont {Bouchaud}},\ }\bibfield  {title} {\bibinfo {title} {Spatial correlations in vote statistics: a diffusive field model for decision-making},\ }\href {https://doi.org/10.1140/epjb/e2010-00151-1} {\bibfield  {journal} {\bibinfo  {journal} {Eur. Phys. J. B}\ }\textbf {\bibinfo {volume} {75}},\ \bibinfo {pages} {395} (\bibinfo {year} {2010})}\BibitemShut {NoStop}%
\bibitem [{\citenamefont {Borghesi}\ \emph {et~al.}(2012)\citenamefont {Borghesi}, \citenamefont {Raynal},\ and\ \citenamefont {Bouchaud}}]{BorRayBou2012}%
  \BibitemOpen
  \bibfield  {author} {\bibinfo {author} {\bibfnamefont {C.}~\bibnamefont {Borghesi}}, \bibinfo {author} {\bibfnamefont {J.-C.}\ \bibnamefont {Raynal}},\ and\ \bibinfo {author} {\bibfnamefont {J.-P.}\ \bibnamefont {Bouchaud}},\ }\bibfield  {title} {\bibinfo {title} {Election turnout statistics in many countries: Similarities, differences, and a diffusive field model for decision-making},\ }\href {https://doi.org/10.1371/journal.pone.0036289} {\bibfield  {journal} {\bibinfo  {journal} {PLOS ONE}\ }\textbf {\bibinfo {volume} {7}},\ \bibinfo {pages} {1} (\bibinfo {year} {2012})}\BibitemShut {NoStop}%
\bibitem [{\citenamefont {Klimek}\ \emph {et~al.}(2012)\citenamefont {Klimek}, \citenamefont {Yegorov}, \citenamefont {Hanel},\ and\ \citenamefont {Thurner}}]{klimek2012statistical}%
  \BibitemOpen
  \bibfield  {author} {\bibinfo {author} {\bibfnamefont {P.}~\bibnamefont {Klimek}}, \bibinfo {author} {\bibfnamefont {Y.}~\bibnamefont {Yegorov}}, \bibinfo {author} {\bibfnamefont {R.}~\bibnamefont {Hanel}},\ and\ \bibinfo {author} {\bibfnamefont {S.}~\bibnamefont {Thurner}},\ }\bibfield  {title} {\bibinfo {title} {Statistical detection of systematic election irregularities},\ }\href@noop {} {\bibfield  {journal} {\bibinfo  {journal} {Proceedings of the National Academy of Sciences}\ }\textbf {\bibinfo {volume} {109}},\ \bibinfo {pages} {16469} (\bibinfo {year} {2012})}\BibitemShut {NoStop}%
\bibitem [{\citenamefont {Jimenez}\ \emph {et~al.}(2017)\citenamefont {Jimenez}, \citenamefont {Hidalgo},\ and\ \citenamefont {Klimek}}]{jimenez2017testing}%
  \BibitemOpen
  \bibfield  {author} {\bibinfo {author} {\bibfnamefont {R.}~\bibnamefont {Jimenez}}, \bibinfo {author} {\bibfnamefont {M.}~\bibnamefont {Hidalgo}},\ and\ \bibinfo {author} {\bibfnamefont {P.}~\bibnamefont {Klimek}},\ }\bibfield  {title} {\bibinfo {title} {Testing for voter rigging in small polling stations},\ }\href@noop {} {\bibfield  {journal} {\bibinfo  {journal} {Science advances}\ }\textbf {\bibinfo {volume} {3}},\ \bibinfo {pages} {e1602363} (\bibinfo {year} {2017})}\BibitemShut {NoStop}%
\bibitem [{\citenamefont {Rozenas}(2017)}]{rozenas2017detecting}%
  \BibitemOpen
  \bibfield  {author} {\bibinfo {author} {\bibfnamefont {A.}~\bibnamefont {Rozenas}},\ }\bibfield  {title} {\bibinfo {title} {Detecting election fraud from irregularities in vote-share distributions},\ }\href@noop {} {\bibfield  {journal} {\bibinfo  {journal} {Political Analysis}\ }\textbf {\bibinfo {volume} {25}},\ \bibinfo {pages} {41} (\bibinfo {year} {2017})}\BibitemShut {NoStop}%
\bibitem [{\citenamefont {Jacobson}(1987)}]{jacobson1987marginals}%
  \BibitemOpen
  \bibfield  {author} {\bibinfo {author} {\bibfnamefont {G.~C.}\ \bibnamefont {Jacobson}},\ }\bibfield  {title} {\bibinfo {title} {The marginals never vanished: Incumbency and competition in elections to the us house of representatives, 1952-82},\ }\href@noop {} {\bibfield  {journal} {\bibinfo  {journal} {American Journal of Political Science}\ ,\ \bibinfo {pages} {126}} (\bibinfo {year} {1987})}\BibitemShut {NoStop}%
\bibitem [{\citenamefont {McCann}(1997)}]{mccann1997threatening}%
  \BibitemOpen
  \bibfield  {author} {\bibinfo {author} {\bibfnamefont {S.~J.}\ \bibnamefont {McCann}},\ }\bibfield  {title} {\bibinfo {title} {Threatening times," strong" presidential popular vote winners, and the victory margin, 1824--1964.},\ }\href@noop {} {\bibfield  {journal} {\bibinfo  {journal} {Journal of personality and social psychology}\ }\textbf {\bibinfo {volume} {73}},\ \bibinfo {pages} {160} (\bibinfo {year} {1997})}\BibitemShut {NoStop}%
\bibitem [{\citenamefont {Mulligan}\ and\ \citenamefont {Hunter}(2003)}]{mulligan2003empirical}%
  \BibitemOpen
  \bibfield  {author} {\bibinfo {author} {\bibfnamefont {C.~B.}\ \bibnamefont {Mulligan}}\ and\ \bibinfo {author} {\bibfnamefont {C.~G.}\ \bibnamefont {Hunter}},\ }\bibfield  {title} {\bibinfo {title} {The empirical frequency of a pivotal vote},\ }\href@noop {} {\bibfield  {journal} {\bibinfo  {journal} {Public Choice}\ }\textbf {\bibinfo {volume} {116}},\ \bibinfo {pages} {31} (\bibinfo {year} {2003})}\BibitemShut {NoStop}%
\bibitem [{\citenamefont {Magrino}\ \emph {et~al.}(2011)\citenamefont {Magrino}, \citenamefont {Rivest},\ and\ \citenamefont {Shen}}]{magrino2011computing}%
  \BibitemOpen
  \bibfield  {author} {\bibinfo {author} {\bibfnamefont {T.~R.}\ \bibnamefont {Magrino}}, \bibinfo {author} {\bibfnamefont {R.~L.}\ \bibnamefont {Rivest}},\ and\ \bibinfo {author} {\bibfnamefont {E.}~\bibnamefont {Shen}},\ }\bibfield  {title} {\bibinfo {title} {Computing the margin of victory in $\{$IRV$\}$ elections},\ }in\ \href@noop {} {\emph {\bibinfo {booktitle} {2011 Electronic Voting Technology Workshop/Workshop on Trustworthy Elections (EVT/WOTE 11)}}}\ (\bibinfo {year} {2011})\BibitemShut {NoStop}%
\bibitem [{\citenamefont {Xia}(2012)}]{xia2012computing}%
  \BibitemOpen
  \bibfield  {author} {\bibinfo {author} {\bibfnamefont {L.}~\bibnamefont {Xia}},\ }\bibfield  {title} {\bibinfo {title} {Computing the margin of victory for various voting rules},\ }in\ \href@noop {} {\emph {\bibinfo {booktitle} {Proceedings of the 13th ACM conference on electronic commerce}}}\ (\bibinfo {year} {2012})\ pp.\ \bibinfo {pages} {982--999}\BibitemShut {NoStop}%
\bibitem [{\citenamefont {Bhattacharyya}\ and\ \citenamefont {Dey}(2021)}]{bhattacharyya2021predicting}%
  \BibitemOpen
  \bibfield  {author} {\bibinfo {author} {\bibfnamefont {A.}~\bibnamefont {Bhattacharyya}}\ and\ \bibinfo {author} {\bibfnamefont {P.}~\bibnamefont {Dey}},\ }\bibfield  {title} {\bibinfo {title} {Predicting winner and estimating margin of victory in elections using sampling},\ }\href@noop {} {\bibfield  {journal} {\bibinfo  {journal} {Artificial Intelligence}\ }\textbf {\bibinfo {volume} {296}},\ \bibinfo {pages} {103476} (\bibinfo {year} {2021})}\BibitemShut {NoStop}%
\bibitem [{\citenamefont {Pal}\ \emph {et~al.}(2024)\citenamefont {Pal}, \citenamefont {Kumar},\ and\ \citenamefont {Santhanam}}]{pal2024universal}%
  \BibitemOpen
  \bibfield  {author} {\bibinfo {author} {\bibfnamefont {R.}~\bibnamefont {Pal}}, \bibinfo {author} {\bibfnamefont {A.}~\bibnamefont {Kumar}},\ and\ \bibinfo {author} {\bibfnamefont {M.}~\bibnamefont {Santhanam}},\ }\bibfield  {title} {\bibinfo {title} {Universal statistics of competition in democratic elections},\ }\href@noop {} {\bibfield  {journal} {\bibinfo  {journal} {arXiv preprint arXiv:2401.05065}\ } (\bibinfo {year} {2024})}\BibitemShut {NoStop}%
\bibitem [{\citenamefont {Brigaldino}(2011)}]{brigaldino2011elections}%
  \BibitemOpen
  \bibfield  {author} {\bibinfo {author} {\bibfnamefont {G.}~\bibnamefont {Brigaldino}},\ }\bibfield  {title} {\bibinfo {title} {Elections in the imperial periphery: Ethiopia hijacked},\ }\href@noop {} {\bibfield  {journal} {\bibinfo  {journal} {Review of African political economy}\ }\textbf {\bibinfo {volume} {38}},\ \bibinfo {pages} {327} (\bibinfo {year} {2011})}\BibitemShut {NoStop}%
\bibitem [{\citenamefont {Frear}(2014)}]{frear2014parliamentary}%
  \BibitemOpen
  \bibfield  {author} {\bibinfo {author} {\bibfnamefont {M.}~\bibnamefont {Frear}},\ }\bibfield  {title} {\bibinfo {title} {The parliamentary elections in belarus, september 2012},\ }\href@noop {} {\bibfield  {journal} {\bibinfo  {journal} {Electoral Studies}\ }\textbf {\bibinfo {volume} {33}},\ \bibinfo {pages} {350} (\bibinfo {year} {2014})}\BibitemShut {NoStop}%
\bibitem [{bel(2020)}]{belarus_report}%
  \BibitemOpen
  \href@noop {} {\bibinfo {title} {Report of organization for security and co-operation in europe (osce)}},\ \bibinfo {howpublished} {\url{https://www.osce.org/odihr/elections/belarus}} (\bibinfo {year} {2020})\BibitemShut {NoStop}%
\bibitem [{\citenamefont {Czwo{\l}ek}\ and\ \citenamefont {Ko{\l}odziejska}(2021)}]{czwolek2021belarusian}%
  \BibitemOpen
  \bibfield  {author} {\bibinfo {author} {\bibfnamefont {A.}~\bibnamefont {Czwo{\l}ek}}\ and\ \bibinfo {author} {\bibfnamefont {J.}~\bibnamefont {Ko{\l}odziejska}},\ }\bibfield  {title} {\bibinfo {title} {Belarusian parliamentary election in 2019},\ }\href@noop {} {\bibfield  {journal} {\bibinfo  {journal} {The Copernicus Journal of Political Studies}\ ,\ \bibinfo {pages} {81}} (\bibinfo {year} {2021})}\BibitemShut {NoStop}%
\bibitem [{\citenamefont {Bedford}(2021)}]{bedford_2021}%
  \BibitemOpen
  \bibfield  {author} {\bibinfo {author} {\bibfnamefont {S.}~\bibnamefont {Bedford}},\ }\bibfield  {title} {\bibinfo {title} {The 2020 presidential election in belarus: Erosion of authoritarian stability and re-politicization of society},\ }\href {https://doi.org/10.1017/nps.2021.33} {\bibfield  {journal} {\bibinfo  {journal} {Nationalities Papers}\ }\textbf {\bibinfo {volume} {49}},\ \bibinfo {pages} {808–819} (\bibinfo {year} {2021})}\BibitemShut {NoStop}%
\bibitem [{ind()}]{india_data}%
  \BibitemOpen
  \href@noop {} {\bibinfo {title} {Election data of india}},\ \bibinfo {howpublished} {\url{https://www.eci.gov.in}}\BibitemShut {NoStop}%
\bibitem [{lok()}]{lokdhaba}%
  \BibitemOpen
  \href@noop {} {\bibinfo {title} {Lok dhaba - a repository of indian election results}},\ \bibinfo {howpublished} {\url{https://lokdhaba.ashoka.edu.in/}}\BibitemShut {NoStop}%
\bibitem [{\citenamefont {Johnston}(2007)}]{johnston2007politics}%
  \BibitemOpen
  \bibfield  {author} {\bibinfo {author} {\bibfnamefont {L.}~\bibnamefont {Johnston}},\ }\href@noop {} {\emph {\bibinfo {title} {Politics: An introduction to the modern democratic state}}}\ (\bibinfo  {publisher} {University of Toronto Press},\ \bibinfo {year} {2007})\BibitemShut {NoStop}%
\bibitem [{\citenamefont {Laakso}\ and\ \citenamefont {Taagepera}(1979)}]{laakso1979effective}%
  \BibitemOpen
  \bibfield  {author} {\bibinfo {author} {\bibfnamefont {M.}~\bibnamefont {Laakso}}\ and\ \bibinfo {author} {\bibfnamefont {R.}~\bibnamefont {Taagepera}},\ }\bibfield  {title} {\bibinfo {title} {“effective” number of parties: a measure with application to west europe},\ }\href@noop {} {\bibfield  {journal} {\bibinfo  {journal} {Comparative political studies}\ }\textbf {\bibinfo {volume} {12}},\ \bibinfo {pages} {3} (\bibinfo {year} {1979})}\BibitemShut {NoStop}%
\bibitem [{\citenamefont {Arnold}\ \emph {et~al.}(2008)\citenamefont {Arnold}, \citenamefont {Balakrishnan},\ and\ \citenamefont {Nagaraja}}]{BarBalNag2008}%
  \BibitemOpen
  \bibfield  {author} {\bibinfo {author} {\bibfnamefont {B.~C.}\ \bibnamefont {Arnold}}, \bibinfo {author} {\bibfnamefont {N.}~\bibnamefont {Balakrishnan}},\ and\ \bibinfo {author} {\bibfnamefont {H.~N.}\ \bibnamefont {Nagaraja}},\ }\href {https://doi.org/10.1137/1.9780898719062} {\emph {\bibinfo {title} {A First Course in Order Statistics}}}\ (\bibinfo  {publisher} {Society for Industrial and Applied Mathematics},\ \bibinfo {year} {2008})\BibitemShut {NoStop}%
\bibitem [{sup()}]{supp}%
  \BibitemOpen
  \href@noop {} {}\bibinfo {note} {See Supplemental Material [URL] for (1) the description of RVM, (2) theoretical calculations for RVM and other related discussions, (3) data summary, and (4) figures.}\BibitemShut {Stop}%
\end{thebibliography}%


\end{document}